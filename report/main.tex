%-----------------------------------------------------------------------------------------
% Final year project report/dissertation template
%-----------------------------------------------------------------------------------------
% R F L Evans (2019)
% Licensed under the public domain (CC0) licence:
%
% The person who associated a work with this deed has dedicated the work to the public 
% domain by waiving all of his or her rights to the work worldwide under copyright law, 
% including all related and neighboring rights, to the extent allowed by law.
%
% You can copy, modify, distribute and perform the work, even for commercial purposes,
%  all without asking permission.
%----------------------------------------------------------------------------------------
%

%-----------------------------------------------------------------------------------------
% Determines the type of document and font size
%-----------------------------------------------------------------------------------------
\documentclass[12pt,a4paper]{article}
%-----------------------------------------------------------------------------------------
% Font control
%-----------------------------------------------------------------------------------------
\usepackage{mathptmx} % Times Roman Font
\usepackage{pythonhighlight}
\usepackage{array}
\usepackage{booktabs}
\usepackage{placeins}

%-----------------------------------------------------------------------------------------
\usepackage{helvet} % Arial/Helvetica font
\renewcommand{\familydefault}{\sfdefault} % Makes serif text all Helvetica
%-----------------------------------------------------------------------------------------
% Set up the page margins
%-----------------------------------------------------------------------------------------
\usepackage[left=2.5cm, right=2.5cm, top=2.5cm]{geometry} % Sets the page margins
%-----------------------------------------------------------------------------------------
% Allow graphics
%-----------------------------------------------------------------------------------------
\usepackage{graphicx}

%-----------------------------------------------------------------------------------------
% Add your report title here
%-----------------------------------------------------------------------------------------
\title{\huge{\textbf{CS528 Final Project: Gas Molecule Simulation}}}

% Add your name here
\author{
	Andrew Dunn\\
	Department of Computer Science\\
	Central Washington University}
\date{\today}

%-----------------------------------------------------------------------------------------
% The start of the document
%-----------------------------------------------------------------------------------------
\begin{document}
	
	%-----------------------------------------------------------------------------------------
	% This adds the title page
	%-----------------------------------------------------------------------------------------
	\maketitle
	\thispagestyle{empty}
	
	\clearpage % moves to the next page
	
	%-----------------------------------------------------------------------------------------
	% This adds the abstract
	%-----------------------------------------------------------------------------------------
	\begin{abstract}
		This project implements a simple simulation of gas particle movement in a closed space using a random walk method. We are representing this space in two dimensions for simplicity.
	\end{abstract}
	\thispagestyle{empty}
	
	%-----------------------------------------------------------------------------------------
	% Move to a new page and set the page numbering from here
	%-----------------------------------------------------------------------------------------
	\clearpage % moves to the next page
	\pagenumbering{arabic}
	
	\section{Introduction} % The start of a new section

	
	
	\section{Methods}\label{methods}

	\begin{python}
	NUM_MOLECULES = 10000
	RANDOM_WALK_DIST = 0.01
	
	# ...
	
	for i in range(NUM_MOLECULES):
		randAngle = np.random.uniform(np.pi * -1, np.pi)
		randVector = np.array([np.cos(randAngle), np.sin(randAngle)])
		randVector = randVector * RANDOM_WALK_DIST
		molX[i] = max(min(molX[i] + randVector[0], 1), 0)
		molY[i] = max(min(molY[i] + randVector[1], 1), 0)
	\end{python}
	
	\begin{python}
	class animatedScatter:
		def __init__(self, molXHist, molYHist):
			self.molXHist = molXHist
			self.molYHist = molYHist
			self.fig = plt.figure()
			self.fig.set_size_inches(PLOT_WIDTH, PLOT_HEIGHT)
			self.ax = plt.axes(xlim=(0, 1), ylim=(0, 1))
			self.itertext = self.ax.text(0.70, 0.9,  '', bbox=dict(facecolor='white', alpha=0.2), transform=self.ax.transAxes)
			print("Creating animation ... Please wait ...")
			self.ani = FuncAnimation(self.fig, self.update, frames=len(molXHist), interval=30, 
			init_func=self.setup, blit=True)
			print("Saving gif ... Please wait ...")
			self.ani.save('final_animated.gif', writer='imagemagick')
			def setup(self):
			self.scatter = self.ax.scatter([], [], color = "green")
			return self.scatter,
		def update(self, i):
			self.itertext.set_text('iteration = %d' % (i * ANIM_FRAMES_TIMESTEP))
			self.scatter.remove()
			self.scatter = self.ax.scatter(self.molXHist[i], self.molYHist[i], color = "green", alpha=0.2)
			return self.scatter,
	\end{python}
	
	\section{Results}\label{results}
	
	\begin{table}[!htb]
		\caption{Experiment functions} 
		\small
		\centering
		\begin{tabular}{ m{5cm} | m{5cm}}
			\toprule
			Number of Molecules & Execution Time (Sec) \\
			\midrule
			100    & 7.5 \\
			1,000  & 74.51 \\
			10,000 & 725.54 \\
			\toprule
		\end{tabular}
		\label{tab:functions}
	\end{table}
	
	\section{Conclusions}\label{conclusions}



	\bibliographystyle{abbrv}
	\bibliography{library}
	
\end{document}